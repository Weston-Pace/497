\documentclass[12pt, letterpaper]{article}
\usepackage[utf8]{inputenc}
\usepackage{graphicx}

\title{Airfoil Analysis}
\author{Weston Pace}
\date{August 2022}

\begin{document}

\maketitle

\section{Potential Flow Code}

This is just a brief look into Potential Flow Code or Panel Code in order to have a better understanding for future readings. 
This method of code is used to determine fluid velocity and pressure distributions sebsequently. Some assumptions of this method are that the fluid is 
inviscid (the viscosity is 0), the fluid is incompressible, the fluid is irrotational, and the variables are unchanging ($\delta/\delta t = 0$)

There are also some boundary conditions that must be met. First, the velocity potential on the internal surface and all points inside V (or on the lower
surface S) is 0. Secondly, the velocity potential on the outer surface is normal to the surface and equal to the freestream velocity.

This is just a quick equation that shows potential flow nicely:

$V = \nabla \phi$ (where $V$ is velocity and $\phi$ is a solution to the method)

A cool thing about potential flow is that multiple solutions to the equation can be added together to form more complex flow models.

There are some limitations to potential flow, though. One is that it doesn't really exist in nature; it is too simplified for most complex situations.
Furthermore, since it is irrotational, it can't be used near solid bodies with boundary layers because vortices will form. 

\section{Thin Airfoil Theory}
Useful definitions:
Chord Line = the imaginary straight line joining the front edge and back edge of an airfoil
Camber = this shows the assymetry of the airfoil about the chord line (takes the average height above or belove the axis with respect to the upper
         and lower components of the airfoil)
Thickness = the distance between the top and lower points of the airfoil
Vortex = A mass of whirling fluid
Freestream Velocity = the velocity of the fluid far ahead of the object before it is affected by the object passing through it
Doublet = A source plus a sink
Coefficient of Drag = the ratio of drag on an object with respect to its velocity and surface area: $c_d = \frac{2F_d}{\rho A_d V_\infty^2}$
Coefficient of Lift = the ratio of lift on an object with respect to its velocity and surface area: $c_d = \frac{2F_l}{\rho A_l V_\infty^2}$
Pitching Moment Coefficient: Relates the ratio of the moment caused be the Lift force (and maybe drag force?) and the wing area, chord length, and 
                             the dynamic pressure: $C_m = \frac{M}{qAc}$
Angle of Attack = The angle formed between a bodies reference line and oncoming flow: $\alpha$
Airfoil Polar = Sometimes a table or graph that shows the relationship between important aerodynamic constants: such as drag and lift coefficients
Lift Curve Slope = As angle of attack changes, the lift (coefficient of lift) changes with a theoretical slope of $2\pi$ (although is is usually a little lower)
Stall = When angle of attack is increased too high past the point of maximum lift, leading to a decrease in lift
Reynold's Number = The ratio of inertial forces to viscous forces of a fluid: $Re = \frac{\rho u L}{\mu}$ (u is the flow speed; L is the characteristic linear dimension;
                    $\mu$ is the dynamic viscosity of the fluid)
Mach Number = The ratio of flow velocity past a boundary to the local speed of sound. More simply, it is a value that shows how many times faster an object is going than the speed of sound: $M$

Equations when integrals using point sources and vortices are used and $y$ approaches $0$:
$q(x)=V_\infty dt/dx$ (where $q$ is the strength of the point sources and $dt/dx$ is the change od the thickness with respect to $x$)
$1/2\pi \int_{0}^{c} \gamma(s)/(x-s) \,ds = V_\infty (\alpha - d\bar{y}/dx)$
$\gamma(\phi) = 2V_\infty\alpha/\tan(\phi) + k/\sin(\phi)$ (where $\gamma$ is the strength of the point vortices and k is some arbitrary constant)

Kutta condition: fixes the problem of discontinuity at the tail edge of the airfoil by making $k = 2V_\infty\alpha$, which leads to:
$\gamma(\phi) = \frac{2V_\infty\alpha}{\sin(\phi)} (1+\cos(\phi))$

Kutta Condition states that the fluid velocity at the tail end of an airfoil must be equal at the top and bottom of the airfoil.

Putting $\gamma$ back in terms of x gives:
$\gamma(x) = 2V_\infty\alpha\frac{\sqrt{1-x/c}}{\sqrt{x/c}}$

Solution for cambered airfoil:
$\gamma(\phi) = 2V_\infty[A_0\frac{1+\cos\phi}{\sin\phi}+\sum_{n} A_n\sin(n\phi)]$
$A_0 = \alpha-1/\pi\int_{0}^{\pi} b(\theta)\, d\theta$
$A_n = 2/\pi\int_{0}^{\pi} b(\theta)\cos(n\theta)\, d\theta$
$b(\theta) = \frac{d\bar{y}}{dx}(x(\theta))$

Equation for a the pressure coefficient in an incompressible, irrotational flow:
$C_p = 1-(V/V_\infty)^2$

Velocity in terms of our components:
$V^2=u^2+v^2=(V_\infty\cos\alpha+u_t+u_c)^2+(V_\infty\sin\alpha+v_t+v_c)^2$ 

Riegel's Correction allows us to fix our high fluid velocity on the leading edge due to the violation of the small disturbance assumption:
$V_{mod}= V\cos\beta$ (where $\beta$ is the slope of the airfoil)
$V_{mod} = V\frac{1}{\sqrt{1+(\frac{d\bar{y}}{dx}\pm\frac{1}{2}\frac{dt}{dx})^2}}$

We can sub in $V_{mod}$ for V in our pressure coefficient equation

We can calculate forces based off the solved pressures:

$F'_x = -p_{u}ds\cos\beta$
$F'_y = p_{u}ds\sin\beta$

Lift and Drag:
$c_l=c_n\cos\alpha -c_a\sin\alpha$
$c_d = c_n\sin\alpha + c_a\cos\alpha$
'n' refers to the normal axis (y), and 'a' refers to the axial axis (x): for a body alligned coordinate system

For Lift:
$\Gamma=V_\infty c\pi[A_0 \frac{A_1}{2}]$
*Kutta-Joukowski Theorem:
$L'=\rho V_\infty \Gamma$

Aerodynamic center is theoretically a quarter of the way down the airfoil

Thin airfoil theory has limitations based off of the small disturbance assumption; this means that at stagnation points, the pressure distribution won't be
well predicted. Inaccuracies may also exist for airfoils of large camber or thickness.

While thin airfoil theory has vortices and sources on the chord line, panel method puts them on the boundary of the airfoil.

Panel method brakes the airfoil into small panels, in which conditions are evaluated over them.

Assumptions of the panel method:
- The source strength is constant over a panel, but it can vary from panel to panel.
- The Vortex strength is constant over the whole airfoil.

Questions for Tuesday:
- What is phi in the equations above
- Why do we approach y = 0 for the integrals
- Why are there different induced velocities based off whether we are looking at the thickness or camber 
- What is the cosine transformation and why does it produce a right triangle with the hypotenuse equal to 1 (e.g. figure 2.19)
- Where does the $b(\theta)$ equation come from
- How many values of n need to be evaluated in the series integral thing
- Ask about Reynold's number (a more in depth explanation)






