\documentclass[12pt, letterpaper]{article}
\usepackage[utf8]{inputenc}
\usepackage{graphicx}

\title{Airfoil Analysis}
\author{Weston Pace}
\date{August 2022}

\begin{document}

\maketitle

\section{Potential Flow Code}

This is just a brief look into Potential Flow Code or Panel Code in order to have a better understanding for future readings. 
This method of code is used to determine fluid velocity and pressure distributions sebsequently. Some assumptions of this method are that the fluid is 
inviscid (the viscosity is 0), the fluid is incompressible, the fluid is irrotational, and the variables are unchanging ($\delta/\delta t = 0$)

There are also some boundary conditions that must be met. First, the velocity potential on the internal surface and all points inside V (or on the lower
surface S) is 0. Secondly, the velocity potential on the outer surface is normal to the surface and equal to the freestream velocity.

This is just a quick equation that shows potential flow nicely:

$V = \nabla \phi$ (where $V$ is velocity and $\phi$ is a solution to the method)

A cool thing about potential flow is that multiple solutions to the equation can be added together to form more complex flow models.

There are some limitations to potential flow, though. One is that it doesn't really exist in nature; it is too simplified for most complex situations.
Furthermore, since it is irrotational, it can't be used near solid bodies with boundary layers because vortices will form. 

\section{Thin Airfoil Theory}
Useful definitions:
Chord Line = the imaginary straight line joining the front edge and back edge of an airfoil
Camber = this shows the assymetry of the airfoil about the chord line (takes the average height above or belove the axis with respect to the upper
         and lower components of the airfoil)
Thickness = the distance between the top and lower points of the airfoil
Vortex = A mass of whirling fluid
Freestream Velocity = the velocity of the fluid far ahead of the object before it is affected by the object passing through it






